\chapter{Summary and Outlook}
The performance of Corryvreckan was analysed and compared to EUTelescope in the scope of this thesis. Furthermore, the reconstruction of low energy proton
tracks with the use of Corryvreckan was investigated.\\
With the track reconstruction of testbeam data, Corryvreckan proved to produce equally valid results in comparison to the EUTelescope framework. No differences
arise in their data conversion and clustering process. A different structure of the tracking and alignment process of the frameworks causes a distinction in the residuals
of the telescope plane. The residual means of the telescope planes are below $\SI{1}{\micro\meter}$ for both softwares, indicating successful telescope alignments.
With standard deviations of $4.450(4)$\,\textmu\, for Corryvreckan and $3.972(3)$\,\textmu\, for EUTelescope, shown for the second sensor plane exemplary, the
reconstrcuted spatial resolutions proved to be well within
the expected value of $\sigma_{\text{M26}} = \SI{18.4}{\micro\meter}/\sqrt{12} \approx \SI{5.311}{\micro\meter}$ for Mimosa26 sensors. The noticebly smaller values
arise due to the used centre-of-gravity algorithm for the calculation of cluster centres. \\
Similar results are also seen for the DUT residuals, shown for the FEI4 sensor in this thesis, as only a third of the RD53a sensor volume was depleted during the testbeam run.
An alignment below $\SI{1}{\micro\meter}$ was achieved for both frameworks with standard deviatons of $21.22(4)$\,\textmu\, and $22.17(4)$\,\textmu\, in horizontal direction and
$72.3(2)$\,\textmu\, and $72.7(2)$\,\textmu\, in vertical direction for Corryvreckan and EUTelescope respectively. With expected spatial resolutions of
$\sigma_{\text{FEI4},\text{x}} \approx \SI{14.434}{\micro\meter}$ and $\sigma_{\text{FEI4},\text{y}} \approx \SI{72.169}{\micro\meter}$ for cluster centres calculated by
arithmetic mean, both frameworks determined a similar, but non optimal value for the horizontal residual of the sensor.
Thus, the results of the reconstruction chain upto the
association of clusters from the DUT planes are similar for the two softwares. Due to the minimization of external dependencies and the more user friendly structure of
Corryvreckan in comparison to EUTelescope, the former is suitable to supersede the EUTelescope framework for track reconstruction in testbeam experiments. \\
