\chapter{Summary and Outlook}
The performance of Corryvreckan was analysed and compared to EUTelescope in the scope of this thesis. Furthermore, the reconstruction of low energy proton
tracks with the use of Corryvreckan was investigated.\\
With the track reconstruction of testbeam data, Corryvreckan proved to produce equally valid results in comparison to the EUTelescope framework. No differences
arise in their data conversion and clustering process. A different structure of the tracking and alignment process of the frameworks causes a distinction in the residuals
of the telescope plane. The residual means of the telescope planes are below $\SI{1}{\micro\meter}$ for both softwares, indicating successful telescope alignments.
With standard deviations of $4.450(4)$\,\textmu m\, for Corryvreckan and $3.972(3)$\,\textmu m\, for EUTelescope, shown for the second sensor plane exemplary, the
reconstrcuted spatial resolutions proved to be well within
the expected value of $\sigma_{\text{M26}} = \SI{18.4}{\micro\meter}/\sqrt{12} \approx \SI{5.311}{\micro\meter}$ for Mimosa26 sensors. The noticebly smaller values
arise due to the used centre-of-gravity algorithm for the calculation of cluster centres. \\
Similar results are also seen for the DUT residuals, shown for the FEI4 sensor in this thesis, as only a third of the RD53a sensor volume was depleted during the testbeam run.
An alignment below $\SI{1}{\micro\meter}$ was achieved for both frameworks with standard deviatons of $21.22(4)$\,\textmu m\, and $22.17(4)$\,\textmu m\, in horizontal direction and
$72.3(2)$\,\textmu m\, and $72.7(2)$\,\textmu m\, in vertical direction for Corryvreckan and EUTelescope respectively. With expected spatial resolutions of
$\sigma_{\text{FEI4},\text{x}} \approx \SI{14.434}{\micro\meter}$ and $\sigma_{\text{FEI4},\text{y}} \approx \SI{72.169}{\micro\meter}$ for cluster centres calculated by
arithmetic mean, both frameworks determined a similar, but non optimal value for the horizontal residual of the sensor. (...) \\
Thus, the results of the reconstruction chain upto the
association of clusters from the DUT planes are similar for the two softwares. Due to the minimization of external dependencies and the more user friendly structure of
Corryvreckan in comparison to EUTelescope, the former is suitable to supersede the EUTelescope framework for track reconstruction in testbeam experiments. \\
Some uses of Corryvreckan, especially alternative modules, were not in the focus of this thesis,
as they were not necessary for further analysis of proton track reconstruction. This includes the determination of the
efficiency of sensors with the AnalysisEfficiency module. Since EUTelescope is not designed to calculate sensor efficiencies, the TBMon2 software was used in the past in addition
to EUTelescope. This opens up the possibility for Corryvreckan to also replace the TBMon2 software, making the track reconstruction and analysis process more efficient.

The reconstruction of simulated protons of different energy with Corryvreckan produced the expected results. Due to no misalignments of sensors in the telescope and uncertainties
in the protons energy, the residuals for high energy protons form sharp peaks. Since, the used IBL sensors have a different spatial resolution, a five-peak structure
forms, which starts to blur into one residual peaks for lower energies below $\SI{1}{\giga\eV}$, due to stronger scattering of the protons. For even lower energies,
around $\SI{200}{\mega\eV}$, the number of reconstructed tracks decrease significantly, as fewer protons deposit energy in all six planes. Since these energies are
used for proton therapy, and a sufficient number of reconstructed protons are necessary to cretae a proton computed tomography image, the patient
will need to be irradiated for longer times to compensate for the small number of reconstructed tracks. However, only small amount of energy is deposited by protons, while traversing
the human body, making longer radiation steps for pct possible without significantly damaging healthy tissue. \\
A greater problematic arises from the high track density of the proton beams causing large track mutiplicities and thus a large number of tracks with false combinations of
clusters. Falsely reconstructed tracks will lower the resolution of the pct image and therefore have to be reduced to a minimum. Using stronger spatial cuts and higher
opening angles for the proton beam increased the ratio of true and false tracks by (...) respectively, while the number of true tracks decreases significantly. Due to the loss of
true tracks for increasingly stricter spatial cuts and larger opening angles, the use of these parameters to improve the ratio is limited for the use of pct. \\
Cuts on track features proved to be useful for the $\chi^2$ value and the horizontal kink angles of the third and fourth plane. The optimal combination of cuts, determined
with a grid search, result in a ratio of 0.477. (...)

With the utilisation of a boosted decision tree used for classification of tracks, the true and false positive rates achieved with the machine learner have better ratios
than any of the previous methods. 
