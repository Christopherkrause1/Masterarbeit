\chapter{Track reconstruction for different energies and track densities}
The quality of the reconstruction of tracks with Corryvreckan depends on numerous factors regarding the particle, the telescope, and the reconstruction software.
For the reconstruction of the proton, the two most crucial properties are their energy, which influences their scatterings as well as the probability of
depositing energy in each sensor, and the particle density, which influences the number of possible combinations of clusters.
Both of these properties significantly affect the number and the quality of the reconstructed tracks, which raises the question, how Corryvreckan performs
under the variation of both of these properties.
In sections \ref{sec:energy} and \ref{sec:density}, the variation of the energy and the track density is investigated respectively.

To analyse the performance of Corryvreckan, the setup from section \ref{sec:setup} is used and tracks are fitted with the general broken lines algorithm from the Tracking4D module.
The proton energies to be analysed are $\SI{50}{\mega\eV}$, $\SI{100}{\mega\eV}$, $\SI{200}{\mega\eV}$
$\SI{1}{\giga\eV}$, $\SI{5}{\giga\eV}$, $\SI{20}{\giga\eV}$ and $\SI{180}{\giga\eV}$. For each value of these parameters a simulation with track densities
of 2, 5, 10, 25, and 50 protons per event and a total amount of 50000 protons is created. \\
The ratio of reconstructed tracks to the total number of tracks for each configuration is shown in table \ref{tab:study}. Here, the relative spatial cut applied is 15.1.


\begin{table}
  \centering
  \begin{tabular}{c | c c c c c c c}
    \toprule
     Protons per event &  $\SI{50}{\mega\eV}$ & $\SI{100}{\mega\eV}$ & $\SI{200}{\mega\eV}$ & $\SI{1}{\giga\eV}$ & $\SI{5}{\giga\eV}$ & $\SI{20}{\giga\eV}$ & $\SI{180}{\giga\eV}$ \\
    \midrule
     2   & 1 & 0.8924 & 0.8224 & 0.8241 & 0.7870 & 0.7790 & 0.7706  \\
     5   & 1 & 0.6328 & 0.5903 & 0.5853 & 0.5184 & 0.5050 & 0.5025  \\
     10  & 0.625 & 0.4038 & 0.3930 & 0.3842 & 0.3224 & 0.3149 & 0.3136  \\
     25  & 0.5 & 0.1823 & 0.1723 & 0.1740 & 0.1441 & 0.1409 & 0.1410  \\
     50  & 0.2727 & 0.0840 & 0.0755 & 0.0788 & 0.0681 & 0.0669 & 0.0674  \\
  \end{tabular}
  \caption{Ratio of true reconstructed tracks to the total amount of reconstructed tracks for different proton energies and protons per event.}
  \label{tab:study}
\end{table}

An increasing track density and energy both cause the ratio to decrease with the track density being the dominant factor influencing this property. Especially for high energies,
the ratios seem to converge, with the lack of simulated uncertainties in the detector setup, the proton energy or the initial direction also contributing to this result.
For a proton energy of $\SI{50}{\mega\eV}$, the number of protons depositing energy is low enough for no false combinatorics to arise. This is the consequence of the
limited statistic of 50000 generated protons.
