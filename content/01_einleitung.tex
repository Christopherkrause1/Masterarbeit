\chapter{Introduction}
The investigation of particles and their properties is the fundament of particle physics. With the development of the standard model in the past century, particles were able to
be classified and their interactions with other particles could be better understood. Particles can be divided into quarks and leptons, making up the matter in the universe,
gauge bosons, which mediate three of the fundamental forces and the higgs boson.

In order to analyse interactions of particles they are accelerated and brought to collision inside of particle colliders. Subsequent particles created in such collisions
can be detected with a surrounding detector system and thus the events can be reconstructed, giving us insight of their behaviour. It is therefore crucial to have
an optimal working detector to analyse of the produced data precisely.

The Large Hadron Collider (LHC) located at CERN is a proton proton collider and currently the largest particle collider, being able to accelerate protons up to $\SI{14}{\tera\eV}$.
It operates since 2008 and helped particle physicists in finding the last missing particle of the standard model, the higgs boson, as well as giving more insight
into properties of particles like the top quark. Standard model parameters can be measured with greater precision, improving our understanding and enables physicists to
test the theory. Physics beyond the standard model is also searched for at the LHC with great interest.

The ATLAS detector is one of the four large experiments at the LHC. % and the largest particle detector with a length of $\SI{46}{\meter}$ and a diameter of $\SI{25}{\meter}$.
It was build to observe many different type of physical phenomena including the detection of the higgs boson in 2012 along with the CMS experiment. In order
to identify the produced particles in the collisions, many layers of different detectors make up the entire ATLAS detector. The most inner layer is the pixel detector, which
is composed of pixel sensor modules to detect particles and reconstruct their tracks.

To measure the efficiency of pixel sensors,
particle tracks are reconstructed in testbeam facilities with a telescope setup made of Mimosa26 pixel sensors to investigate the devices under test (DUT), which are
placed inside the telescope. These measurements are necessary to ensure an optimal performance of sensor modules inside the experiments.
