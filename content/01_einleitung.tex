\chapter{Introduction}
The investigation of particles and their properties is the foundation of particle physics. With the development of the standard model in the past century, particles
and their interactions with other particles could be better understood. The fundamental particles can be divided into quarks and leptons, making up the matter in the universe,
gauge bosons, which mediate three of the fundamental forces, and the Higgs boson.

In order to analyse interactions of particles, they are accelerated and brought to collision inside of particle colliders. The particles created in such collisions
can be detected with a surrounding detector system and thus the events can be reconstructed, giving an insight into their properties. It is, therefore, crucial to have
an optimal working detector to analyse the produced data precisely.
Currently, the largest collider is
the Large Hadron Collider (LHC) \cite{lhc} located at CERN, which is able to accelerate protons up to
a centre of mass energy of $\SI{14}{\tera\eV}$.
It operates since 2008 and helped particle physicists in finding the last missing particle of the standard model, the Higgs boson \cite{higgs}\cite{higgs_cms}, as well as giving more insight
into properties of particles like the top quark. Standard model parameters can be measured with greater precision, which improves the understanding of fundamental physical processes
and enables physicists to test their theories. %Physics beyond the standard model is also searched for at the LHC with great interest.

Besides the use of detectors in high-energy physics experiments, they are also a crucial part of medical equipment, specifically used for radiative treatments. \\
The idea to use protons as an effective particle for the treatment of cancer arose in 1946 \cite{1946} and over the years it proved to have several advantages over
conventional x-ray radiation therapy. To apply proton therapy, a radiation plan has to be created to determine the location of the tumor and the dose of the radiation.
For that purpose tomography scans create images of the inside of patients, enabling the precise irradiation of tumors.
The benefits of using protons for tomography scans in comparison to x-ray tomography scans for proton therapy caused a growing interest in this technique \cite{pbt}. \\
In order to create images with proton computed tomography, the reconstruction of proton tracks traversing the human body and the measurement of their energy is necessary.

This thesis focuses on the reconstruction of proton tracks under conditions present in proton computed tomography scans. To achieve that, the new track reconstruction software
Corryvreckan \cite{corryvreckan} is analysed based on its capability of reconstructing tracks in comparison to the EUTelescope framework \cite{gbl} known
for its long use in detector physics. Furthermore,
it is used to reconstruct
proton tracks, which are generated in Allpix$^2$ \cite{allpix} simulations to investigate the possibility of making proton computed tomography scans feasible with this framework.


%An interest for the imaging
%technique of proton computed tomogrophy came up around 1990, with proton therapy gaining in popularity due to its advantages over conventional x-ray therapy.
%In order to create images with proton computed tomography, the reconstruction of proton tracks traversing the human body is


%The ATLAS detector is one of the four large experiments at the LHC. % and the largest particle detector with a length of $\SI{46}{\meter}$ and a diameter of $\SI{25}{\meter}$.
%It was build to observe many different type of physical phenomena including the detection of the higgs boson in 2012 along with the CMS experiment. In order
%to identify the produced particles in the collisions, many layers of different detectors make up the entire ATLAS detector. The most inner layer is the pixel detector, which
%is composed of pixel sensor modules to detect particles and reconstruct their tracks.
%
%To measure the efficiency of pixel sensors,
%particle tracks are reconstructed in testbeam facilities with a telescope setup made of Mimosa26 pixel sensors to investigate the devices under test (DUT), which are
%placed inside the telescope. These measurements are necessary to ensure an optimal performance of sensor modules inside the experiments.
