\thispagestyle{plain}

\section*{Kurzfassung}
Die Protonentherapie ist, wegen ihrer höheren Präzision der Energiedeposition, eine wirkungsvolle Alternative zu herkömmlichen Röntgentherapie.
Auftretende Unsicherheiten durch die Verwendung von Röntgen Computertomographie können mit dem Verfahren der Protonen Computertomographie reduziert werden.
Um damit einen Bestrahlungsplan anfertigen zu können ist es nötig die Spuren der Protonen durch den Körper mithile von Detektoren zu rekonstruieren. \\
In dieser Arbeit wird die neue Spurrekonstruktions Software Corryvreckan mit der in Zukunft eingestellten Software EUTelescope auf ihre Funktionalität im
Rekonstruieren von Spuren in Testbeam Experimenten untersucht. Es wird gezeigt, dass Corryvreckan in der Lage ist Spuren mit vergleichbarer Qualität zu rekonstruieren.\\
Des Weiteren werden Simulation von Protonstrahlen durch ein Teleskop erstellt, um die rekonstruierbarkeit von niederenergetischen Protonen hoher Dichte
mit Corryvreckan zu analysieren. Die Verwendung eines machinellen Lerners zur Klassifizierung von rekonstruierten Spuren
erzielt hierbei bessere Ergebnisse als die Anwendung von räumlichen Schnitten und Schnitte auf relevante Spureigenschaften.

\section*{Abstract}
\begin{english}
Proton therapy is, because of its higher precision of energy deposition,
an effective alternative to conventional X-ray therapy. Occurring
uncertainties due to the use of x-ray computed tomography can be
be reduced with the procedure of proton computed tomography. To
make an irradiation plan, it is necessary to reconstruct the tracks of the protons
through the body with the help of detectors. \\
In this work the new track reconstruction software Corryvreckan is compared with the soon discontinued
software EUTelescope for its functionality in reconstructing tracks in test beams. 
It is shown that
Corryvreckan is able to reconstruct tracks with comparable quality. \\
Furthermore, simulations of proton beams through a telescope are created,
to analyse the reconstructability of low energy high density proton with Corryvreckan. The use of a machine learner for the
classification of reconstructed tracks achieves better results than the application of spatial cuts and cuts to relevant track properties.


\end{english}
