\thispagestyle{plain}

\section*{Kurzfassung}
Die Protonentherapie ist wegen ihrer höheren Präzision der Energiedeposition eine wirkungsvolle Alternative zur herkömmlichen Röntgentherapie.
Auftretende Unsicherheiten durch die Verwendung von Röntgen Computertomographie für die Bestrahlungsplanung der Protonentherapie können mit dem Verfahren der Protonen Computertomographie reduziert werden.
Um damit einen Bestrahlungsplan anfertigen zu können, ist es nötig, die Spuren der Protonen durch den Körper mithile von Detektoren zu rekonstruieren. \\
In dieser Arbeit wird die neue Spurrekonstruktionssoftware Corryvreckan mit der in Zukunft eingestellten Software EUTelescope auf ihre Funktionalität im
Rekonstruieren von Spuren in Testbeam Experimenten verglichen. Es wird gezeigt, dass Corryvreckan in der Lage ist, Spuren mit vergleichbarer Qualität zu rekonstruieren.\\
Des Weiteren werden Simulationen von Protonstrahlen durch ein Teleskop erstellt, um die Leistung von Corryvreckan beim Rekonstruieren von niederenergetischen
Protonenstrahlen mit hoher Dichte zu analysieren. Hohe Teilchendichten führen bei der Rekonstruktion zu Spuren mit falschen Clusterkombinationen.
Die Verwendung eines maschinellen Lerners zur Unterscheidung von richtig und falsch rekonstruierten Spuren
erzielt hierbei bessere Ergebnisse als die Anpassung von Parametern im \mbox{Rekonstruktionsalgorithmus}.
%Anwendung von räumlichen Schnitten und Schnitte auf relevante Spureigenschaften.

\section*{Abstract}
\begin{english}
Proton therapy is, due to its higher precision of energy deposition,
an effective alternative to conventional x-ray therapy. Occurring
uncertainties due to the use of x-ray computed tomography for radiation planning of proton therapy can be
be reduced with the technique of proton computed tomography. To
make an irradiation plan, it is necessary to reconstruct the tracks of the protons
through the body with the help of detectors. \\
In this work the new track reconstruction software Corryvreckan is compared with the soon discontinued
software EUTelescope for its functionality in reconstructing tracks in beam test.
It is shown that
Corryvreckan is able to reconstruct tracks with comparable quality. \\
Furthermore, simulations of proton beams through a telescope are created,
to analyse the performance of the reconstruction of low energy high density proton beams with Corryvreckan.
High particle densities lead to tracks with incorrect cluster combinations during reconstruction.
The use of a machine learner
to differentiate between true and false reconstructed tracks achieves better results than adjusting parameters in the reconstruction algorithm.


\end{english}
