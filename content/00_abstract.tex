\thispagestyle{plain}

\section*{Kurzfassung}
Die Protonentherapie ist, wegen ihrer höheren Präzision der Energiedeposition, eine wirkungsvolle Alternative zu herkömmlichen Röntgentherapie.
Auftretende Unsicherheiten durch die Verwendung von Röntgen Computertomographie können mit dem Verfahren der Protonen Computertomographie reduziert werden.
Um damit einen Bestrahlungsplan anfertigen zu können ist es nötig die Spuren der Protonen durch den Körper mithile von Detektoren zu rekonstruieren. \\
In dieser Arbeit wird die neue Spurrekonstruktions Software Corryvreckan mit der in Zukunft eingestellten Software EUTelescope auf ihre Funktionalität im
Rekonstruieren von Spuren in Testbeam Experimenten untersucht. Es wird gezeigt, dass Corryvreckan in der Lage ist Spuren mit vergleichbarer Qualität zu rekonstruieren.\\
Des Weiteren werden Simulation von Protonstrahlen durch ein Teleskop erstellt, um die rekonstruierbarkeit von niederenergetischen Protonen
mit Corryvreckan zu analysieren. (Hier noch Ergebnis)

\section*{Abstract}
\begin{english}




\end{english}
